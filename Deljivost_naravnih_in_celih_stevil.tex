\section{Deljivost naravnih in celih števil}

\begin{frame}
    \sectionpage
\end{frame}

\begin{frame}
    \tableofcontents[currentsection, hideothersubsections]
\end{frame}

    \subsection{Relacija deljivosti}

        \begin{frame}
            \frametitle{Relacija deljivosti}
        \end{frame}

    \subsection{Kriteriji deljivosti}

        \begin{frame}
            \frametitle{Kriteriji deljivosti}
        \end{frame}

    \subsection{Praštevila in sestavljena števila}

        \begin{frame}
            \frametitle{Praštevila in sestavljena števila}
        \end{frame}

    \subsection{Osnovni izrek o deljenju}

        \begin{frame}
            \frametitle{Osnovni izrek o deljenju}
        \end{frame}

    \subsection{Največji skupni delitelj in najmanjši skupni večkratnik}

        \begin{frame}
            \frametitle{Največji skupni delitelj in najmanjši skupni večkratnik}
        \end{frame}

        \begin{frame}[t]
            \frametitle{Najmanjši skupni večkratnik}

            % \Large\textbf{Najmanjši skupni večkratnik števil}
            % ~\\
            % \normalsize

            \begin{alertblock}{}
                \textbf{Najmanjši skupni večkratnik} števil $a$ in $b$ je najmanjše število od tistih, ki so deljiva s številoma $a$ in $b$. \\ 
                Oznaka: $\mathbf{v(a,b)}$.
            \end{alertblock}

            \pause

            \begin{block}{Izračun $v(a,b)$}
                \begin{itemize}
                    \item Števili $a$ in $b$ prafaktoriziramo;
                    \item iz prafaktorizacij vzamemo vse različne potence praštevil na največji eksponent.
                \end{itemize}
            \end{block}
        \end{frame}

        \begin{frame}[t]
            \Large\textbf{Najmanjši skupni večkratnik izrazov}
            ~\\
            \normalsize

            \pause

            \begin{alertblock}{}
                \textbf{Najmanjši skupni večkratnik} izrazov je tak izraz, ki je deljiv z vsemi izrazi, ki nastopajo. 
            \end{alertblock}

            \pause

            \begin{block}{Kako določimo najmanjši skupni večkratnik izrazov?}
                \begin{itemize}
                    \item<4-> Izraze razstavimo;
                    \item<5-> vzamemo vse faktorje, ki nastopajo;
                    \item<6-> če je kateri izmed faktorjev potenciran, vzamemo njegovo potenco z največjim eksponentom.
                \end{itemize}
            \end{block}
        \end{frame}

        \begin{frame}[t]
            \begin{exampleblock}{Naloga}
                Določite najmanjši skupni večkratnik izrazov:
                \begin{itemize}
                    \item<1-> $4a^2b^3$ in $6a^3b$;
                    \item<2-> $12m^5n^7$, $18m^2n^8$ in $m^4n^5$;
                    \item<3-> $5x^3yz^{12}$, $13w^2xy^7z$ in $65wy^5z^2$;
                    \item<4-> $x^{n+1}+x^n$ in $2x^{n+1}$;
                    \item<5-> $25-x^2$ in $x^2-10x+25$;
                    \item<6-> $8x^2-24x-32$, $20x^2+40x+20$ in $10x^2+10$;
                    \item<7-> $6x^2+24x$, $6x^3-96x$ in $6x^4+48x^3+96x^2$;
                    \item<8-> $x^3+x^2+x+1$, $4x^2+8x+4$ in $x^3+1$.
                \end{itemize}
            \end{exampleblock}
        \end{frame}

