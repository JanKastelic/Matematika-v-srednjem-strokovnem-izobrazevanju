\documentclass[1pt, aspectratio=169]{beamer}

\usepackage[T1]{fontenc}
\usepackage[utf8]{inputenc}
\usepackage[slovene]{babel}
\usepackage{lmodern}
\usepackage{amsfonts,amssymb,amsmath}
\usepackage{pgfpages}
% \usepackage{tikz}
\usepackage{wrapfig}
\usepackage{graphicx}
\usepackage{pgfkeys}
\usepackage{pgfplots}
\usepackage{xcolor}
\usepackage{tkz-euclide}
\usepackage{xfp}
% \usepackage{pgf}

\pgfplotsset{compat=1.18} 

\usetikzlibrary{angles,arrows,arrows.meta,calc,decorations,decorations.markings,decorations.pathreplacing,decorations.shapes,decorations.text,
	decorations.pathmorphing,intersections,math,plotmarks,positioning,quotes,shapes.misc,through}



% \setbeameroption{show notes on second screen}
% \setbeameroption{show only notes}

% \usetheme[sectionpage=simple, titlestyle=plain, sectionstyle=style2, slidestyle=style1, numbering=counter, block=fill, headingcolor=theme]{trigon}

\usetheme{CambridgeUS}
\usecolortheme{beaver}



\setbeamerfont{subtitle}{size=\small}


\title{MATEMATIKA}
\subtitle{2. letnik -- srednje strokovno izobraževanje}
\date{\today}
\author{Jan Kastelic}
\institute[FMF]{Fakulteta za matematiko in fiziko, \\ Univerza v Ljubljani}

\newtheorem{izrek}{Izrek}
\newcommand{\Vir}[1]{\color{gray}{\tiny{Vir: #1}}}


\begin{document}

\begin{frame}
	\titlepage
\end{frame}
	
% \titleframe

\begin{frame}
	\frametitle{Vsebina}
	\tableofcontents[hideallsubsections]
\end{frame}

\section{Geometrija v ravnini}

\begin{frame}
    \sectionpage
\end{frame}

\begin{frame}
    \tableofcontents[currentsection, hideothersubsections]
\end{frame}

    \subsection{Osnovni pojmi}

        \begin{frame}
            \frametitle{Osnovni pojmi}
        \end{frame}

    \subsection{Skladnost in merjenje}

        \begin{frame}
            \frametitle{Skladnost in merjenje}
        \end{frame}

    \subsection{Vzporednost in pravokotnost}

        \begin{frame}
            \frametitle{Vzporednost in pravokotnost}
        \end{frame}

    \subsection{Trikotnik}

        \begin{frame}
            \frametitle{Trikotnik}
        \end{frame}

    \subsection{Krožnica, krog, lok}

        \begin{frame}
            \frametitle{Krožnica, krog, lok}
        \end{frame}

    \subsection{Štirikotnik in pravilni $n$-kotnik}

        \begin{frame}
            \frametitle{Štirikotnik in pravilni $n$-kotnik}
        \end{frame}

    \subsection{Podobnost}

        \begin{frame}
            \frametitle{Podobnost}
        \end{frame}

    \subsection{Izreki v pravokotnem trikotniku}

        \begin{frame}
            \frametitle{Izreki v pravokotnem trikotniku}
        \end{frame}

    \subsection{Kotne funkcije}

        \begin{frame}
            \frametitle{Kotne funkcije}
        \end{frame}

	
\section{Metrična geometrija v ravnini}

\begin{frame}
    \sectionpage
\end{frame}

\begin{frame}
    \tableofcontents[currentsection, hideothersubsections]
\end{frame}

    \subsection{Ploščina in obseg}

        \begin{frame}
            \frametitle{Ploščina in obseg}
        \end{frame}

    \subsection{Razreševanje trikotnika}

        \begin{frame}
            \frametitle{Razreševanje trikotnika}
        \end{frame}

    \subsection{Korg}

        \begin{frame}
            \frametitle{Krog}
        \end{frame}


\section{Potence in koreni}

\begin{frame}
    \sectionpage
\end{frame}

\begin{frame}
    \tableofcontents[currentsection, hideothersubsections]
\end{frame}

    \subsection{Potence z naravnimi in celimi eksponenti}

        \begin{frame}
            \frametitle{Potence z naravnimi in celimi eksponenti}
        \end{frame}

    \subsection{Potence z racionalnimi eksponenti}

        \begin{frame}
            \frametitle{Potence z racionalnimi eksponenti}
        \end{frame}

    \subsection{Kvadratni in kubični koren}

        \begin{frame}
            \frametitle{Kvadratni in kubični koren}
        \end{frame}

    \subsection{Koreni poljubnih stopenj}

        \begin{frame}
            \frametitle{Koreni poljubnih stopenj}
        \end{frame}

    \subsection{Iracionalna enačba}

        \begin{frame}
            \frametitle{Iracionalna enačba}
        \end{frame}



\section{Funkcija}

\begin{frame}
    \sectionpage
\end{frame}

\begin{frame}
    \tableofcontents[currentsection, hideothersubsections]
\end{frame}

    \subsection{Funkcija in njene lastnosti}

        \begin{frame}
            \frametitle{Funkcija in njene lastnosti}
        \end{frame}

    \subsection{Potenčna funkcija}

        \begin{frame}
            \frametitle{Potenčna funkcija}
        \end{frame}

    \subsection{Kvadratna funkcija}

        \begin{frame}
            \frametitle{Kvadratna funkcija}
        \end{frame}



\section{(i) Vektorji}

\begin{frame}
    \sectionpage
\end{frame}

\begin{frame}
    \tableofcontents[currentsection, hideothersubsections]
\end{frame}

    \subsection{Definicija vektorja}

        \begin{frame}
            \frametitle{Definicija vektorja}
        \end{frame}

    \subsection{Vzporedni premik}

        \begin{frame}
            \frametitle{Vzporedni premik}
        \end{frame}

    \subsection{Seštevanje in odštevanje vektorjev}

        \begin{frame}
            \frametitle{Seštevanje in odštevanje vektorjev}
        \end{frame}

    \subsection{Množenje vektorja s številom}

        \begin{frame}
            \frametitle{Množenje vektorja s številom}
        \end{frame}

    \subsection{Kolinearnost in komplanarnost vektorjev}

        \begin{frame}
            \frametitle{Kolinearnost in komplanarnost vektorjev}
        \end{frame}

    \subsection{Pravokotni koordinatni sistem v prostoru}

        \begin{frame}
            \frametitle{Pravokotni koordinatni sistem v prostoru}
        \end{frame}

    \subsection{Vektorji v pravokotnem koordinatnem sistemu}

        \begin{frame}
            \frametitle{Vektorji v pravokotnem koordinatnem sistemu}
        \end{frame}

    \subsection{Skalarni produkt}

        \begin{frame}
            \frametitle{Skalarni produkt}
        \end{frame}

    \subsection{Skalarni produkt v pravokotnem koordinatnem sistemu}

        \begin{frame}
            \frametitle{Skalarni produkt v pravokotnem koordinatnem sistemu}
        \end{frame}


\section{Kompleksna števila}

\begin{frame}
    \sectionpage
\end{frame}

\begin{frame}
    \tableofcontents[currentsection, hideothersubsections]
\end{frame}

    \subsection{Množica kompleksnih števil}

        \begin{frame}
            \frametitle{Množica kompleksnih števil}
        \end{frame}

    \subsection{Računanje s kompleksnimi števili}

        \begin{frame}
            \frametitle{Računanje s kompleksnimi števili}
        \end{frame}

    \subsection{Konjugiranje kompleksnega števila}

        \begin{frame}
            \frametitle{Konjugiranje kompleksnega števila}
        \end{frame}

    \subsection{Deljenje kompleksnih števil}

        \begin{frame}
            \frametitle{Deljenje kompleksnih števil}
        \end{frame}

    \subsection{Absolutna vrednost kompleksnega števila}

        \begin{frame}
            \frametitle{Absolutna vrednost kompleksnega števila}
        \end{frame}

    \subsection{Kvadratna enačba in kompleksna števila}

        \begin{frame}
            \frametitle{Kvadratna enačba in kompleksna števila}
        \end{frame}



\end{document}