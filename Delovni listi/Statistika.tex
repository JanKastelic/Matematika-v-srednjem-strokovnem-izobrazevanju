\documentclass[12pt,a4paper]{article}
\usepackage[utf8]{inputenc}
\usepackage[T1]{fontenc}
\usepackage[slovene]{babel}
\usepackage{lmodern}
\usepackage{enumitem}
\usepackage{amsmath}
\usepackage{amssymb}
\usepackage{float} 
\usepackage{url} 
%\usepackage{hyperref} 
\usepackage{graphicx} 
\usepackage[margin=0.3in]{geometry}
\usepackage{multirow}

\newcommand{\code}{\texttt}

\title{Statistika v Excel-u}
\author{}
\date{\today}

\begin{document}

\maketitle

V datoteki \code{Rezultati-izpita-neizpolnjeno.xlsx} so zbrani podatki o desežkih študentov pri enem izmed predmetov. 
Na podalgi danega niza podatkov rešite spodnje naloge z uporabo vgrajenih funkcij. Na listu \textit{Podatki} so že predpripravljene razpredelnice za reševanje. 
Vse grafikone, ki jih boste oblikovali, premaknite na nov list, in ga smiselno poimenujte.

\begin{enumerate}
    \item Izračunajte število vseh študentov, ki so bili vpisani k predmetu. Pri tem si pomagajte s funkcijo \code{COUNT}.
    \item Poiščite ustrezne vrednosti o doseženih točkah iz stolpca \textit{Izpit + domače naloge}. Uporabite funkcije \code{AVERAGE, MODE, MEDIAN, MAX, MIN}.
    \item Vnaprej so vam podani razredi podatkov in opisi le-teh. Izračunajte pripadajoče frekvence in velikosti središččnih kotov, ki bi posameznem razredu pripadle, ob risanju strukturnega kroga. V pomoč naj  vam bo funkcija \code{COUNTIF}.
    \item Oblikujte grafikon (strukturni krog), ki bo prikazoval število posameznih končnih dosežkov študentov.
    \item Oblikujte še grafični prikaz (vrstični) števila študentov glede na doseženo oceno, ki so pri predmetu opravljali izpit.
    \item Določite povprečno oceno in oceno, ki je bila med študenti pridobljena največkrat (uporabite zgolj podatke o pozitivnih ocenah). V pomoč naj vam bosta funkciji \code{MODE, AVERAGE}.
    \item Podatke 1. izpitnega roka grupirajte po razredih 'standardne' ocenjevalne lestvice, ki velja v srednji šoli (1--5). Določite širine in sredine razredov ter njihove frekvence. Pri tem uporabite funkcijo \code{FREQUENCY}.
    \item Za podatke o končnem številu pridobljenih točk določite vrednosti kvartilov. Tu je uporabna funkcija \code{QUARTILE}.
    \item Na istem grafikonnu oblikujte 'škatli z brki' za podatke o zbranih točkah 1. in 2. izpitnega roka.
\end{enumerate}


\end{document}