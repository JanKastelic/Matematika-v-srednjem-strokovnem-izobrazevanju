\documentclass[12pt, aspectratio=169]{beamer}

\usepackage[T1]{fontenc}
\usepackage[utf8]{inputenc}
\usepackage[slovene]{babel}
\usepackage{lmodern}
\usepackage{amsfonts,amssymb,amsmath}
\usepackage{pgfpages}
% \usepackage{tikz}
\usepackage{wrapfig}
\usepackage{graphicx}
\usepackage{pgfkeys}
\usepackage{pgfplots}
\usepackage{xcolor}
\usepackage{tkz-euclide}
\usepackage{xfp}
% \usepackage{pgf}

\pgfplotsset{compat=1.18} 

\usetikzlibrary{angles,arrows,arrows.meta,calc,decorations,decorations.markings,decorations.pathreplacing,decorations.shapes,decorations.text,
	decorations.pathmorphing,intersections,math,plotmarks,positioning,quotes,shapes.misc,through}



% \setbeameroption{show notes on second screen}
% \setbeameroption{show only notes}

% \usetheme[sectionpage=simple, titlestyle=plain, sectionstyle=style2, slidestyle=style1, numbering=counter, block=fill, headingcolor=theme]{trigon}

\usetheme{CambridgeUS}
\usecolortheme{beaver}



\setbeamerfont{subtitle}{size=\small}


\title{MATEMATIKA}
\subtitle{1. letnik -- srednje strokovno izobraževanje}
\date{\today}
\author{Jan Kastelic}
\institute[FMF]{Fakulteta za matematiko in fiziko, \\ Univerza v Ljubljani}

\newtheorem{izrek}{Izrek}
\newcommand{\Vir}[1]{\color{gray}{\tiny{Vir: #1}}}


\begin{document}

\begin{frame}
	\titlepage
\end{frame}
	
% \titleframe

\begin{frame}
	\frametitle{Vsebina}
	\tableofcontents[hideallsubsections]
\end{frame}
	
\section{Naravna in cela števila}

\begin{frame}
    \sectionpage
\end{frame}

\begin{frame}
    \tableofcontents[currentsection, hideothersubsections]
\end{frame}

    \subsection{naravna števila}

        \begin{frame}
            \frametitle{Naravna števila}
        \end{frame}

    \subsection{Cela števila}

        \begin{frame}
            \frametitle{Cela števila}
        \end{frame}

    \subsection{Urejenost naravnih in celih števil}

        \begin{frame}
            \frametitle{Urejenost naravnih in celih števil}
        \end{frame}

    \subsection{Potence z naravnimmi eksponenti}

        \begin{frame}
            \frametitle{Potence z naravnimi eksponenti}
        \end{frame}

    \subsection{Večkratniki in izrazi}

        \begin{frame}
            \frametitle{Večkratniki in izrazi}
        \end{frame}


\section{Deljivost naravnih in celih števil}

\begin{frame}
    \sectionpage
\end{frame}

\begin{frame}
    \tableofcontents[currentsection, hideothersubsections]
\end{frame}

    \subsection{Relacija deljivosti}

        \begin{frame}
            \frametitle{Relacija deljivosti}
        \end{frame}

    \subsection{Kriteriji deljivosti}

        \begin{frame}
            \frametitle{Kriteriji deljivosti}
        \end{frame}

    \subsection{Praštevila in sestavljena števila}

        \begin{frame}
            \frametitle{Praštevila in sestavljena števila}
        \end{frame}

    \subsection{Osnovni izrek o deljenju}

        \begin{frame}
            \frametitle{Osnovni izrek o deljenju}
        \end{frame}

    \subsection{Največji skupni delitelj in najmanjši skupni večkratnik}

        \begin{frame}
            \frametitle{Največji skupni delitelj in najmanjši skupni večkratnik}
        \end{frame}

        \begin{frame}[t]
            \frametitle{Najmanjši skupni večkratnik}

            % \Large\textbf{Najmanjši skupni večkratnik števil}
            % ~\\
            % \normalsize

            \begin{alertblock}{}
                \textbf{Najmanjši skupni večkratnik} števil $a$ in $b$ je najmanjše število od tistih, ki so deljiva s številoma $a$ in $b$. \\ 
                Oznaka: $\mathbf{v(a,b)}$.
            \end{alertblock}

            \pause

            \begin{block}{Izračun $v(a,b)$}
                \begin{itemize}
                    \item Števili $a$ in $b$ prafaktoriziramo;
                    \item iz prafaktorizacij vzamemo vse različne potence praštevil na največji eksponent.
                \end{itemize}
            \end{block}
        \end{frame}

        \begin{frame}[t]
            \Large\textbf{Najmanjši skupni večkratnik izrazov}
            ~\\
            \normalsize

            \pause

            \begin{alertblock}{}
                \textbf{Najmanjši skupni večkratnik} izrazov je tak izraz, ki je deljiv z vsemi izrazi, ki nastopajo. 
            \end{alertblock}

            \pause

            \begin{block}{Kako določimo najmanjši skupni večkratnik izrazov?}
                \begin{itemize}
                    \item<4-> Izraze razstavimo;
                    \item<5-> vzamemo vse faktorje, ki nastopajo;
                    \item<6-> če je kateri izmed faktorjev potenciran, vzamemo njegovo potenco z največjim eksponentom.
                \end{itemize}
            \end{block}
        \end{frame}

        \begin{frame}[t]
            \begin{exampleblock}{Naloga}
                Določite najmanjši skupni večkratnik izrazov:
                \begin{itemize}
                    \item<1-> $4a^2b^3$ in $6a^3b$;
                    \item<2-> $12m^5n^7$, $18m^2n^8$ in $m^4n^5$;
                    \item<3-> $5x^3yz^{12}$, $13w^2xy^7z$ in $65wy^5z^2$;
                    \item<4-> $x^{n+1}+x^n$ in $2x^{n+1}$;
                    \item<5-> $25-x^2$ in $x^2-10x+25$;
                    \item<6-> $8x^2-24x-32$, $20x^2+40x+20$ in $10x^2+10$;
                    \item<7-> $6x^2+24x$, $6x^3-96x$ in $6x^4+48x^3+96x^2$;
                    \item<8-> $x^3+x^2+x+1$, $4x^2+8x+4$ in $x^3+1$.
                \end{itemize}
            \end{exampleblock}
        \end{frame}



\section{Osnove logike in teorije množic}

\begin{frame}
    \sectionpage
\end{frame}

\begin{frame}
    \tableofcontents[currentsection, hideothersubsections]
\end{frame}

    \subsection{Izjave in izjavne povezave}

        \begin{frame}
            \frametitle{Izjave in izjavne povezave}
        \end{frame}

    \subsection{(i) Množice in računanje z njimi}

        \begin{frame}
            \frametitle{(i) Množice in računanje z njimi}
        \end{frame}


\section{Racionalna števila}

\begin{frame}
    \sectionpage
\end{frame}

\begin{frame}
    \tableofcontents[currentsection, hideothersubsections]
\end{frame}

    \subsection{Ulomki}

        \begin{frame}
            \frametitle{Ulomki}
        \end{frame}

    \subsection{Računanje z ulomki}

        \begin{frame}
            \frametitle{Računanje z ulomki}
        \end{frame}

    \subsection{Potence s celimi eksponenti}

        \begin{frame}
            \frametitle{Potence s celimi eksponenti}
        \end{frame}

    \subsection{Ulomki in decimalni zapis}

        \begin{frame}
            \frametitle{Ulomki in decimalni zapis}
        \end{frame}

    \subsection{Algebrski ulomki}

        \begin{frame}
            \frametitle{Algebrski ulomki}
        \end{frame}



\section{Realna števila}

\begin{frame}
    \sectionpage
\end{frame}

\begin{frame}
    \tableofcontents[currentsection, hideothersubsections]
\end{frame}

    \subsection{Množica realnih števil}

        \begin{frame}
            \frametitle{Množica realnih števil}
        \end{frame}

    \subsection{Kvadratni in kubični koren}

        \begin{frame}
            \frametitle{Kvadratni in kubični koren}
        \end{frame}

    \subsection{Interval}

        \begin{frame}
            \frametitle{Interval}
        \end{frame}

    \subsection{Linearna enačba}

        \begin{frame}
            \frametitle{Linearna enačba}
        \end{frame}

    \subsection{Sistem dveh linearnih enačb}

        \begin{frame}
            \frametitle{Sistem dveh linearnih enačb}
        \end{frame}

    \subsection{Sistem treh in več linearnih enačb}

        \begin{frame}
            \frametitle{Sistem treh in več linearnih enačb}
        \end{frame}

    \subsection{Razmerje, sorazmerje}

        \begin{frame}
            \frametitle{Razmerje, sorazmerje}
        \end{frame}

    \subsection{Procentni račun}

        \begin{frame}
            \frametitle{Procentni račun}
        \end{frame}

    \subsection{Linearna neenačba}

        \begin{frame}
            \frametitle{Linearna neenačba}
        \end{frame}

    \subsection{Absolutna vrednost}
        
        \begin{frame}
            \frametitle{Absolutna vrednost}
        \end{frame}

    \subsection{Zaokroževanje, približki in napake}

        \begin{frame}
            \frametitle{Zaokroževanje, približki in napake}
        \end{frame}


\section{Linearna funkcija}

\begin{frame}
    \sectionpage
\end{frame}

\begin{frame}
    \tableofcontents[currentsection, hideothersubsections]
\end{frame}

    \subsection{Koordinatni sistem}

        \begin{frame}
            \frametitle{Koordinatni sistem}
        \end{frame}

    \subsection{Razdalja med dvema točkama}

        \begin{frame}
            \frametitle{Razdalja med dvema točkama}
        \end{frame}

    \subsection{(i) Obseg in ploščina trikotnika}

        \begin{frame}
            \frametitle{(i) Obseg in ploščina trikotnika}
        \end{frame}

    \subsection{Funkcija in njene lastnosti}

        \begin{frame}
            \frametitle{Funkcija in njene lastnosti}
        \end{frame}

    \subsection{Linearna funkcija}

        \begin{frame}
            \frametitle{Linearna funkcija}
        \end{frame}

    \subsection{Enačbe premice}

        \begin{frame}
            \frametitle{Enačbe premice}
        \end{frame}

    \subsection{Modeliranje z linearno funkcijo}

        \begin{frame}
            \frametitle{Modeliranje z linearno funkcijo}
        \end{frame}


\section{Osnove statistike}

\begin{frame}
    \sectionpage
\end{frame}

\begin{frame}
    \tableofcontents[currentsection, hideothersubsections]
\end{frame}

    \subsection{Osnovni pojmi statistike}

        \begin{frame}
            \frametitle{Osnovni pojmi statistike}
        \end{frame}

    \subsection{Urejanje in grupiranje podatkov}

        \begin{frame}
            \frametitle{Urejanje in grupiranje podatkov}
        \end{frame}

    \subsection{Grafično prikazovanje podatkov}

        \begin{frame}
            \frametitle{Grafično prikazovanje podatkov}
        \end{frame}

    \subsection{Srednje vrednosti}

        \begin{frame}
            \frametitle{Srednje vrednosti}
        \end{frame}

    \subsection{Razpršenost}

        \begin{frame}
            \frametitle{Razpršenost}
        \end{frame}



\end{document}